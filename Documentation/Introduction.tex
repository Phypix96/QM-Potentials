\documentclass[12pt,twoside,a4paper]{book}
\usepackage[utf8]{inputenc}
\usepackage{amsmath}
\usepackage{mathtools}
\usepackage{hyperref}
\usepackage{float}
\usepackage{multirow}
\usepackage{amsfonts}
\usepackage{siunitx}
\usepackage{qtree}

\begin{document}

\frontmatter

\title{Documentation of "QM-Potentials"}
\author{Philipp Haim}
\maketitle

\chapter{Introduction}
Bei dem Projekt ging es primär nicht darum, eine M\"{o}glichkeit zu haben, verschiedene Potentialt\"{o}pfe
darzustellen, sondern vielmehr um die Erfahrung, die es bei einem solchen Projekt zu sammeln gibt.
Aus demselben Grund wurde dieser Text geschrieben, um Erfahrung bei dem Umgang mit Latex zu bekommen.


\mainmatter

\chapter{Quamtummechanical Theory}
In this chapter, the physical properties of the various potentials will be discussed, in oder to 


\begin{itemize}
\item Infinite Potential Well
\item Finite Potential Well
\item Harmonic Potential
\item Coulomb Potential
\end{itemize}

\section{Infinite Potential Well}\label{infinite}

\section{Finite Potential Well}\label{finite}

\end{document}